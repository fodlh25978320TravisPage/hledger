% hledger-Sh.tex
\begin{hcarentry}[updated]{hledger}
\label{hledger}
\report{Simon Michael}%14/05
\status{ongoing development; suitable for daily use}
\makeheader

hledger is a library and end-user tool (with command-line, curses and web
interfaces) for converting, recording, and analyzing financial
transactions, using a simple human-editable plain text file format. It is
a haskell port and friendly fork of John Wiegley's Ledger, licensed under
GNU GPLv3+.

%% 2011:
%% hledger aims to be a reliable, practical tool for daily use. It reports
%% charts of accounts or account balances, filters transactions by type,
%% helps you record new transactions, converts CSV data from your bank,
%% publishes your text journal with a rich web interface, generates simple
%% charts, and provides an API for use in your own financial scripts and
%% apps.

%% In the last six months there have been two major releases. 0.15 focussed
%% on features and 0.16 focussed on quality. Changes include:

%% - new modal command-line interface, extensible with hledger-* executables in the path
%% - more useful web interface, with real account registers and basic charts
%% - hledger-web no longer needs to create support files, and uses latest yesod & warp
%% - more ledger compatibility
%% - misc command enhancements, API improvements, bug fixes, documentation updates
%% - lines of code increased by 3k to 8k
%% - project committers increased by 6 to 21

%% Current plans include:

%% - Continue the release rhythm of odd-numbered = features, even-numbered =
%%   quality/stability/polish, and releasing on the first of a month

%% - In 0.17, clean up the storage layer, allow rcs integration via
%%   filestore, and read (or convert) more formats

%% - Keep working towards wider usefulness, improving the web interface and
%%   providing standard financial reports

\FurtherReading
  \url{http://hledger.org}
\end{hcarentry}

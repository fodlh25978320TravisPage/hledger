% hledger-Sh.tex
\documentclass[DIV16,twocolumn,10pt]{scrreprt}
\usepackage{paralist}
\usepackage{graphicx}
\usepackage[final]{hcar}

%include polycode.fmt

\begin{document}

\begin{hcarentry}{hledger}
\report{Simon Michael}
\status{(active development)}
%% \participants{(PARTICIPANTS OTHER THAN MYSELF)}% optional
\makeheader

%% (WHAT IS IT?)

hledger is a cross-platform program (and Haskell library) for tracking
money, time, or any other commodity, using double-entry accounting and
a simple, editable text file format.  
hledger aims to be a reliable, practical tool for daily use,
and provides command-line, curses-style, and web interfaces.
It is inspired by and largely compatible with John Wiegley's Ledger program.  
hledger is released under GNU GPLv3+.

%% (WHAT IS ITS STATUS? / WHAT HAS HAPPENED SINCE LAST TIME?)

hledger's HCAR entry was last updated in the November 2011 report, but
development has continued steadily, with 2-3 major releases each year.

Many new features and improvements have been introduced, making hledger much more useful.
These include:
\begin{compactitem}
\item Easier installation, using stack, system packages, or downloadable Windows binaries.
\item A simpler and more robust web interface, with built-in help, balance charts, flexible transaction entry, and automatic browser startup
\item A new curses-style interface, hledger-ui, is now included and fully supported (in 0.27)
\item The command-line interface is more robust, and is aware of terminal width, COLUMNS, and wide characters (0.27)
\item New commands: accounts, balancesheet, cashflow, incomestatement
\item New add-on packages: ledger-autosync, hledger-diff, hledger-interest, and hledger-irr
\item hledger can now report current value based on market prices (-V, in 0.27)
\item The journal format has become richer, supporting more Ledger features such as balance assertions
\item hledger journals and reports can be exported as CSV
\item hledger now reads CSV files directly, using flexible conversion rules
\item The balance command can show multiple columns, with per-period changes or ending balances
\item Depth-limiting now interacts well with other features, making it effective for summarising
\item hledger-web's query language is richer and is also used by the command-line interface
\item The Decimal library is used for representing amounts exactly
\item Unicode is handled correctly
\item Many commands are faster
\end{compactitem}

Project updates include:
\begin{compactitem}
\item hledger.org and the docs have been refreshed a few times, and now include many examples
\item hledger's code repo and bug tracker have moved from darcs/darcs hub/google code to git/github
\item hledger has its own IRC channel on freenode: #hledger, with logging and commit/issue/travis notifications
\end{compactitem}

%% (CAN OTHERS GET IT?)

hledger is available from hledger.org, github, hackage, stackage, and
is packaged for a number of systems including Debian, Ubuntu, Gentoo,
Fedora, and NixOS.  See \url{http://hledger.org/download} or
\url{http://hledger.org/developer-guide} for guidance.

%% (WHAT ARE THE IMMEDIATE PLANS?)

Immediate plans: ship hledger 0.27, then: 
improve parser speed and memory efficiency, 
integrate a separate parser for Ledger files built by John Wiegley,
and work towards the 1.0 release.

\FurtherReading
  \url{http://hledger.org}
\end{hcarentry}

\end{document}
